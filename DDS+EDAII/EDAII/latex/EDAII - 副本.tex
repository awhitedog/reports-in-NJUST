\documentclass[12pt]{article}

\usepackage[UTF8]{ctex}
\usepackage{appendix}
\usepackage{enumerate}
\usepackage{amsmath}
\usepackage{graphicx}
\usepackage{cite}
\usepackage{ listings} 
\usepackage{bigstrut}
\usepackage{multirow}
\usepackage{geometry}
\usepackage{longtable}
\usepackage{listings}
\usepackage{xcolor}
\usepackage{color}
\usepackage[section]{placeins}
\usepackage[colorlinks,linkcolor=blue]{hyperref}
\usepackage[]{caption2} 
\renewcommand{\captionlabeldelim}{ }
\usepackage{titlesec}  
\usepackage{titletoc}
\geometry{a4paper,scale=0.8}
\renewcommand\figurename{\heiti\zihao{5} 图}
\renewcommand\tablename{\heiti\zihao{5} 表}
\renewcommand{\contentsname}{\centering 目录}
\lstset{
    columns=fixed,       
    numbers=left,                                        % 在左侧显示行号
    frame=none,                                          % 不显示背景边框
    backgroundcolor=\color[RGB]{245,245,244},            % 设定背景颜色
    keywordstyle=\color[RGB]{40,40,255}\bf,                 % 设定关键字颜色
    numberstyle=\footnotesize\color{darkgray},           % 设定行号格式
    commentstyle=\it\color[RGB]{0,96,96},                % 设置代码注释的格式
    stringstyle=\rmfamily\slshape\color[RGB]{128,0,0},   % 设置字符串格式
%labelstyle=\tiny\color[RGB]{255,0,0},
    showstringspaces=false,                              % 不显示字符串中的空格
    escapeinside=``,
    language=VHDL,                                        % 设置语言
}
\begin{document}%文档从这里开始。
\renewcommand\refname{参考文献}

\section{总计时模块}
\subsubsection{由模7、模24、模60模块组成的基本计时模块}
利用VHDL语言的IF语句,对计数的条件进行判断。例如模7,则在计数到0111(7)时跳转到0001,即可实现数字1到7的循环计数。
\begin{lstlisting}[ language=VHDL,]
library ieee;
use ieee.std_logic_1164.all;
use ieee.std_logic_unsigned.all;
ENTITY mo7 IS 
 PORT
( en   :IN  std_logic;
clear:IN  std_logic;--when 0,output is always 0
clk  :IN  std_logic;
ql :buffer std_logic_vector(3 downto 0)
);
END mo7;
ARCHITECTURE beh OF mo7 IS
BEGIN
PROCESS(clk,clear)
BEGIN
IF(clear='0')THEN
ql<="0000";
ELSIF(clk'EVENT AND clk='1')THEN
if(en='1')then
if(ql=7) then
ql<="0001";
else
ql<=ql+1;
end if;
end if;
END IF;
END PROCESS;
END beh;
} 
\end{lstlisting} 
\end{document}
