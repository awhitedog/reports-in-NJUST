\documentclass[12pt]{article}% 这是主要的格式。

\usepackage[UTF8]{ctex}
\usepackage{appendix}
\usepackage{enumerate}
\usepackage{amsmath}
\usepackage{graphicx}
\usepackage{cite}
\usepackage{geometry}
\usepackage[section]{placeins}
\usepackage[colorlinks,linkcolor=blue]{hyperref}
\title{穿林海跨雪原气冲霄汉,抒豪情寄壮志面对群山}
\author{——《林海雪原》(人民文学出版社2012年,曲波著)}
%\author{}
\date{9161040G0734 许晓明}
\geometry{scale=0.75}

\begin{document}%文档从这里开始。



\maketitle
“三十六颗红心向太阳,破风雪如闪电奔驰山岗,披星戴月越战越强,经得起风,受得起浪”。这是现代京剧《智取威虎山》中的选段,唱词借少剑波之口,表达了东北民主联军小分队的壮志豪情。如今读罢曲波《林海雪原》原本,便又是另一番更深层次的感受了。

《林海雪原》英雄无数,最为人知晓的便是“打虎上山”的杨子荣。他年纪轻轻,却深谋远虑,屡次立功,信念坚定。献礼先遣图、智识小炉匠、盛布酒肉兵、活捉座山雕……其足智多谋可见一斑。智谋之外,杨子荣亦英勇非常,龙潭虎穴,他敢去闯一闯;身处险境,他能随机应变。

再如少剑波,如小分队的其他人一样,穿山风的狂卷,密集的枪弹雨林,都吓不退他。作为队长“203”,他率小分队进驻夹皮沟,发动群众组织生产自救,组建民兵队伍。而他与小分队护士“白鸽”白茹的爱情,吸引眼球之余,更显出角色血肉。

此外,还有骁勇威猛、谋略不足的刘勋苍,身怀绝技、粗俗诙谐的栾超家,忠厚老实、刻苦耐劳的“长腿”孙达得,青春年少、不幸殒命的高波等等。英雄们个个性格鲜明,传奇经历也不重复,以致读罢掩卷,脑子里留下了个个鲜活的印象。

但《林海雪原》讲述的又不仅是解放战争初期剿匪斗争的惊鸿一瞥,它所表现出的更是一种智慧,一股勇气,一份信念,融合人性之美,成为一个独特的缩影。对战士们来说,死,并不可怕。比死重要的事还有很多。他们为保卫自己深爱的土地,为保卫可亲可爱的乡亲们,为保卫伟大的祖国而血染雪原,甚至葬身林海。他们心中永远有一个信念,那便是——真正地解放人民。他们执着的追求着人生目标,即使前方有枪林弹雨,有无数艰难险阻,也义不容辞,勇往直前。

《林海雪原》投射的这种缩影,带领着读者进入一片蓝天。尽管有时,乌云会阻挡阳光,但在一阵甘霖的挥洒后,清流依旧,白鸽挂着风铃远翔,叶更绿,天更蓝,阳光在露珠上闪烁。世界循环不停,美不停。任他竹冷松寒,笑谈一壶浊酒。

\end{document}
