\documentclass[12pt]{article}% 这是主要的格式。

\usepackage[UTF8]{ctex}
\usepackage{appendix}
\usepackage{enumerate}
\usepackage{amsmath}
\usepackage{graphicx}
\usepackage{cite}
\usepackage{geometry}
\usepackage[section]{placeins}
\usepackage[colorlinks,linkcolor=blue]{hyperref}
\title{从非物质文化遗产及相关民间传统文化传承及传播情况看传统文化的现代生命力——以南京市为例}
\author{9161040G0734 许晓明(李羊城班), 9161040G0737 袁善帅(李羊城班),\\9161040G0739 周宏伟(李羊城班)\&9161040G0741 邹嘉璇(李羊城班)}
%\author{}
\date{}
\geometry{left=2.0cm,right=2.0cm}

\begin{document}%文档从这里开始。

\numberwithin{footnote}{section}
\renewcommand{\contentsname}{\centering 目录}
\renewcommand{\tablename}{表}
\renewcommand{\figurename}{图}
\renewcommand\refname{参考文献}
\renewcommand\appendix{\setcounter{secnumdepth}{0}}
\renewcommand\abstractname{摘要}



\maketitle

\abstract{非物质文化遗产增强了人们对文化多样性和人类创造力的尊重。而2017年3月1日实施的《南京市非物质文化遗产保护条例》则正式使南京市非物质文化遗产保护工作进入了新阶段。但非物质文化遗产及相关民间传统文化在传承及传播过程中仍然存在一些问题,民众对非遗及相关民间传统文化的认知度还不是很高。伴随着民众对非遗及相关民间传统文化自豪感与认同感的提高,我们可以从中窥探传统文化的现代生命力。本文通过问卷调查的形式,分析了南京市非物质文化遗产及相关民间传统文化传承和发展上的困境,总结了造成这些困境的原因并给出了相应的建议。}
\\

\textbf{关键词}:非物质文化遗产\ 民间\ 传统文化\ 传承\ 传播\ 南京市

\tableofcontents

\section{导言}

\subsection{非物质文化遗产历史沿革概要}

非物质文化遗产的概念最早始于20 世纪50年代的日本。1989年,联合国科教文组织以《保护传统文化和民俗的建议》的形式指出,“政府在保护民俗中应扮演一个决定性的角色,并尽快采取行动”
。1998年,《宣布人类口头和非物质遗产代表作条例》(以下简称《条例》)提出的“口头和非物质遗产”一词
\footnote{指“来自某一文化社区的全部创作,这些创作以传统为依据、由某一群体或一些个体所表达并被认为是符合社区期望的作为其文化和社会特性的表达形式;其准则和价值通过模仿或其他方式口头相传,它的形式包括:语言、文学、音乐、舞蹈、游戏、神话、礼仪、习惯、手工艺、建筑术及其他艺术。”}
成为后来“非物质文化遗产”的前身;《条例》还“鼓励各国政府、各非政府组织和各地方社区开展鉴别、保护和利用其口头和非物质遗产的活动,因为这种遗产是各国人民集体记忆的保管者,只有它能够确保文化特性永存”
,为非物质文化遗产保护的重大意义提供了依据。

2003年,联合国科教文组织正式通过《保护非物质文化遗产公约(2003)》(以下简称《公约》),将非物质文化遗产定义为:“被各社区、群体,有时是个人,视为其文化遗产组成部分的各种社会实践、观念表述、表现形式、知识、技能以及相关的工具、实物、手工艺品和文化场所。”
按照这一定义,《公约》给出了非物质文化遗产的常见形式,包括以下方面:
 \begin{enumerate}[ (a).]
\item 口头传统和表现形式,包括作为非物质文化遗产媒介的语言;
\item 表演艺术;
\item 社会实践、仪式、节庆活动;
\item 有关自然界和宇宙的知识和实践;
\item 传统手工艺。
\end{enumerate}
这为世界各国保护非物质文化遗产奠定了良好的基础;《公约》还强调,“这种非物质文化遗产世代相传,在各社区和群体适应周围环境以及与自然和历史的互动中,被不断地再创造,为这些社区和群体提供认同感和持续感,从而增强对文化多样性和人类创造力的尊重。”

据此,2004年4月,文化部、财政部联合发出《关于实施中国民族民间文化保护工程的通知》
。而国务院于2005年3月下发的《关于加强我国非物质文化遗产保护工作的意见》
,则正式拉开了我国全面保护“非物质文化遗产”的时代大幕。

自2006第一批国家级非物质文化遗产名录公布至今,一大批“非遗”从民间走出。2011年2月,第十一届全国人民代表大会常务委员会第十九次会议通过《中华人民共和国非物质文化遗产法》(以下简称《法案》),以期以法律的形式加强对非物质文化遗产的保护。《法案》将非物质文化遗产定义为:“各族人民世代相传并视为其文化遗产组成部分的各种传统文化表现形式,以及与传统文化表现形式相关的实物和场所。”。包括:
    \begin{enumerate}[ (a).]
      \item 传统口头文学以及作为其载体的语言;
      \item 传统美术、书法、音乐、舞蹈、戏剧、曲艺和杂技;
      \item 传统技艺、医药和历法;
      \item 传统礼仪、节庆等民俗;
      \item 传统体育和游艺;
      \item 其他非物质文化遗产。
    \end{enumerate}
\subsection{南京市非物质文化遗产情况简述}

“江南佳丽地,金陵帝王州”。南京这座具有2500年悠久历史的文化名城,作为六朝古都,十代国都、拥有着中国四大古都之一的美誉,它不仅是中华文明的重要发祥地之一,也是是国务院首批确定的历史文化名城,其悠久的历史,灿烂的文化,以及劳动人民的智慧结晶,不仅创造了丰富的物质财富,而且创造了丰富的精神财富,形成了特殊的非物质文化遗产。目前,南京市现有非物质文化遗产项目2000余个
\footnote{
截至2014年第四批国家级非物质文化遗产代表性项目名录时。
}
。%从南京的非物质文化遗产项目分类来看,主要有以下几种类型:

%一是民间文学、民间传说,是指由劳动人民直接创造的、在民间广泛流传的文学,主要是口头文学,如神话、传说、民间故事、民间戏曲、民间曲艺、歌谣等。南京民间传说的最显著的特点在于它的历史性、地方性、人民性和情节性。目前流传下来的南京传说主要有:《董永传说》、《项羽与浦口的历史传说》、《长芦民间传说》、《卞和献玉传说》、《崔致远与双女坟的 故事》、《脱尾龙传说》、《伍子胥和浣纱女的故事》、《后羿射日》等。

%二是民间音乐,指形成并流行于民间的歌曲和器乐曲,还包括民间舞蹈音乐和民间戏曲音乐。它具有鲜明的民族风格和地方特色,过去主要通过口头的方式流传。南京的民间音乐主要包括古琴艺术(金陵琴派)、留左吹打乐、六合民歌鲜花调、高淳民歌、马铺锣鼓等。

%三是民间舞蹈,它产生和流传于民间、风格鲜明、为广大群众喜闻乐见的舞蹈,反映了人民的劳动、斗争、交际和爱情生活,有集体性、自由性、娱乐性、普及性和健身性的特点,它把民间最纯朴的舞蹈和民间的风情,加以美化夸张但不失民间的味道。南京最典型的民间舞蹈有竹马(东坝大马灯)、龙舞(骆山大龙)、江浦手狮、麻雀蹦、跳五猖、栖霞龙舞、打社火、跳当当、铜山高台狮子舞、长芦抬龙、骨牌灯、茶山会、砖墙打罗 汉、阳江打水浒、龙吟车、沛桥高跷、小马灯等。

%四是传统戏剧,指以语言、动作、舞蹈、音乐、木偶等形式达到叙事目的的舞台表演艺术的总称,大致可分为大戏、小戏与偶戏等三种。南京地方戏剧最有代表性的是阳腔目连戏、洪山戏、皮影戏等。

%五是传统曲艺,曲艺是中华民族各种说唱艺术的统称,它是由民间口头文学和歌唱艺术经过长期发展演变形成的一种独特的艺术形式。流传在南京的曲艺主要有南京白局、南京评话、送春、南京白话、打五件、送麒麟等。

%六是传统体育、游艺与杂技。传统体育是指我国各民族传统的祛病、健身、习武和娱乐活动项目。传统体育和民族传统舞蹈共生共通,融合发展,将成为一种文化发展的趋势,同时也是构成人们余暇生活的一个重要内容。南京老百姓在长期的生活实践中形成了自己独特的传统体育与游艺杂技项目, 如:抖空竹、六月六龙舟竞渡、殷巷石锁等。

%七是民间美术,是人民群众创作的,以美化环境、丰富民间风俗活动为目的,在日常生活中应用、流行的美术。现今在南京流传的民间美术主要为:南京剪纸、秦淮灯彩、十竹斋饾彩拱花技艺、南京仿古牙雕、南京仿古木雕、金陵竹刻、周岗红木雕刻、南京微雕、戏剧脸谱、南京瓷刻等。

%八是传统手工技艺,指具有高度技巧性、艺术性的手工,如挑花、刺绣、缂丝等。由于南京的历史悠久,遗留并传承下来的手工技艺种类较多,较为典型的有南京云锦木机妆花手工织造技艺、南京金箔锻造技艺、金陵刻经印刷技艺、金银细工制作技艺、南京板鸭、盐水鸭制作工艺、真金线制作技 艺、高淳羽毛扇制作技艺、绿柳居素菜烹制技艺、刘长兴面点加工制作技艺、金陵折扇制作技艺、雨花茶炒制工艺、方山裱画技艺、 窦村石刻技艺、南京钟制造工艺、秦淮风味小吃加工制作技艺、龙袍蟹黄汤包加工制作技艺、明觉铁画锻制技艺等。

%九是传统医药,它是指与古代社会文化密切相联的医学实践,其实践因不同地区的传统文化继承性的差别而显示出多样化。目前在南京留传的传统医药有张简斋国医医术、高淳梁氏骨科、灵芝传统知识及应用等。

%十是民俗,即民间风俗,指一个地区或民族中广大民众所创造、享用和传承的生活文化。南京的民俗具有浓厚的地方色彩。主要有秦淮灯会、妈祖庙会、南京赏梅习俗、雨花石鉴赏习 俗、蒋王庙庙会、南京祠山庙会、泰山庙会、狮子岭庙会、牛首山踏春习俗、薛城花台会、夫子庙花鸟鱼虫市、上梁仪式、 水八鲜饮食习俗、南京老地名等。




在已经被认定的各级非物质文化遗产代表性项目当中,古琴艺术·金陵琴派-The Guqin and its Music、南京云锦织造技艺-The craftsmanship of Nanjing Yunjin brocade、剪纸·南京剪纸-Chinese paper-cut、中国雕版印刷技艺·金陵刻经印刷技艺-China engraved block printing technique等4 个项目被联合国教科文组织列为人类非物质文化遗产代表性项目,南京云锦木机妆花手工织造技艺、南京金箔锻制技艺、秦淮灯会、南京白局、金陵刻经印刷技艺、竹马·东坝大马灯、古琴艺术·金陵琴派、龙舞·骆山大龙、剪纸·南京剪纸、灯彩·秦淮灯彩、金银细工制作技艺等11个项目被列为国家级代表性项目,75个项目被列为省级保护名录,145个项目被确定为市级保护名录,13 人被确认为国家级非物质文化遗产代表性项目代表性传承人。
\subsection{问题的提出}

既然南京市有着如此辉煌的非物质文化遗产资源,那么南京市民及其他地区居民对南京市非物质文化遗产及相关民间传统文化的认知度、认同感、自豪感又是否与这座城市相称?随着物质文明和精神文明的发展,作为精神文明建设的一部分补充内容,这些非物质文化遗产在人们看来面临着哪些挑战与机遇?在2017年3月1日,《南京市非物质文化遗产保护条例》正式实施的背景下,非遗的保护与发扬情况究竟是什么局面呢?

对非物质文化遗产及相关民间传统文化的保护与抢救,其目的与归宿应当是使其回归民间、回归生活,而不是进入博物馆。因此,非物质文化遗产及相关民间传统文化能否在民间生存,便成为了检验其保护成果的重要依据之一。为此,我们进行了相关调查。

\subsection{文献综述}

目前,非遗保护工作看似具备完整的保护构架,但是在现实中非遗的保护依旧存在较多漏洞。石冰倩等在南京云锦保护的研究中指出,南京传统手工技艺的传承人数量十分有限,推广途经也较为单一。由于传承人中年轻人占少数,在传承与革新方面遇到了极大的阻碍。付启元在探寻非遗利用途径时提出,在经济转轨、社会转型的新形势下,南京非物质文化面临外来文化的冲击和现代文明的挑战,生存环境受到威胁。陈燕等则强调,必须科学地对南京的非物质文化遗产进行经营。刘舟在南京金箔工艺的保护研究中,提到了非遗所具有的历史与人文内涵消失的问题。陆亚华等则从对外宣传的角度出发,指出南京网络外宣仍然存在内容贫乏,形式单一等问题。王春雩在分析了南京白局艺术特点时,发现南京的一部分曲艺文化因受众的局限性、社会的关注度不足等原因面临绝种的困境。

但这些文献资料,大多是以政府或传承人的角度进行探究,而忽略了民众对非物质文化遗产及民间传统文化的认知度、自豪感情况。非遗是民间的非遗,脱离了群众的非遗是畸形的非遗,只有让非物质文化遗产回归民间、回归生活才是保护非物质文化遗产的目的与归宿。基于此,我们进行了进一步的调查。


\section{南京市“非遗”与民间传统文化情况问卷调查及调查对象相关情况}

本次调查采用问卷调查的方式,于2018年1-2月在网络上发布问卷。其中,设计的问卷包括三部分,第一部分是调查对象的基本情况;第二部分是调查对象对一些南京市非物质文化遗产及相关民间传统文化与其载体的了解程度;第三部分是南京市非物质文化遗产及相关传统文化的发展现状、前景及建议的民间性认知调查。问卷设有单选题和多选题。发放问卷304份,其中有效问卷286份,有效回收率为94.07\%。调查样本中,性别的比例较为均匀;年龄的比例中,18-25岁(大学生群体)的调查对象占了绝大多数,18岁以下(年轻群体)所占比例为次。25-35岁、35岁以上(中老年群体)的调查对象较少;在地区分布上,在调查一定量的南京市民的基础上,也采集其他地区的相关数据,由于网络问卷调查的特殊性,对于受调查对象的地区难以把握
,其他地区的调查对象数量较多,但仍保证了有一定数量的南京市民参与调查问卷。调查对象的基本情况具体如表1-3所示。

从表中的相关情况可知,本次问卷调查一定程度上可以大致反映以青年群体为主的、南京及部分其他地区人民大众对南京市非物质文化遗产的现状认知情况。

\begin{table}[htbp]
  \centering
   \caption{\bf{调查对象性别情况表}}
    \begin{tabular}{ccc}
    \hline
    性别    & 人数    & 比例 \\
    \hline
    男     & 137   & 47.90\% \\
    女     & 149   & 52.10\% \\
    \hline
    \end{tabular}
    \end{table}%

\begin{table}[htbp]
\centering
  \caption{\bf{调查对象年龄分布情况表}}
\begin{tabular}{ccc}
\hline
年龄    & 人数    & 比例 \\
\hline
18岁以下 & 47    & 16.43\% \\
18-25岁 & 197   & 68.88\% \\
25-35岁 & 23    & 8.04\% \\
35岁以上 & 19    & 6.64\% \\
\hline
\end{tabular}%
\end{table}%


\begin{table}[htbp]
\centering
\caption{\bf{调查对象所处地区情况表}}
\begin{tabular}{ccc}
\hline
地区    & 人数    & 比例 \\
\hline
南京市   & 102   & 35.66\% \\
其他地区  & 184   & 64.34\% \\
\hline
\end{tabular}%
\end{table}%


\section{问卷调查结果}

考虑到一部分的非物质文化遗产,其继承、发扬及其相关载体与本地方言密不可分的关系,在受调查的南京市民中,我们单独开设一问- 您是否会说南京话?- 以期大致推测南京市非物质文化遗产及传统文化的语言类继承、发扬情况,其结果如表4所示。

\begin{table}[htbp]
  \centering
  \caption{\bf{调查对象中南京市民能否说本地方言情况表}}
    \begin{tabular}{ccc}
    \hline
    选项    & 人数    & 比例 \\
    \hline
    会     & 69    & 67.65\% \\
    不会    & 33    & 32.35\% \\
    \hline
    \end{tabular}%  \label{tab:addlabel}%
\end{table}%

从表中的相关数据中,我们可以看出,目前南京(青少年)市民中,能够说本地方言的市民数量仍然占大多数;与之对应的,也有超过3 成的市民不再能够说本地方言了。由于样本数量的关系,南京市的方言遗失情况可能并不完全尽然,但在一定程度上也大致可以说明,目前南京市有一定程度的方言遗失。同时我们也可以由此预见,南京市非物质文化遗产中与方言相关联的部分内容的保护工作,将面临较为严峻的形势。


\subsection{南京市部分非物质文化遗产的民间了解情况}

在目前南京市拥有的非物质文化遗产及相关民间传统文化中,我们选择了传统曲艺文化,民间舞蹈艺术,传统手工技艺,南京美食秦淮八绝\footnote{
指永和园黄桥烧饼与开洋干丝、蒋有记牛肉汤与牛肉锅贴、六凤居豆腐涝与葱油饼、奇芳阁鸭油酥烧饼与什锦菜包、奇芳阁麻油素干丝与鸡丝浇面、莲湖糕团店桂花夹心小元宵与五色小糕、瞻园面馆熏鱼银丝面与薄皮包饺、魁光阁五香豆与五香蛋等八套秦淮风味小吃。
}等进行民间性认知的问卷调查。在我们选取的调查项目中,它们或者是具有一定的文化意义、属于国家级或省级非物质文化遗产代表名录的,或者是濒临失传、亟待抢救性保护的,或者是与市民生活息息相关的。其民间性了解情况如表5-8所示。



\begin{table}[htbp]
  \centering
  \caption{\bf{市民对南京市传统曲艺的了解情况}}
    \begin{tabular}{ccc}
    \hline
    传统曲艺文化  & 了解人数  & 比例 \\
        \hline
    白局    & 52    & 18.10\% \\
    白话    & 52    & 18.10\% \\
    评话    & 44    & 15.52\% \\
    阳腔目连戏 & 15    & 5.17\% \\
    洪山戏   & 18    & 6.45\% \\
    都不了解  & 164   & 57.34\% \\
        \hline
    \end{tabular}%
\end{table}%

\begin{table}[htbp]
  \centering
  \caption{\bf{市民对南京市民间舞蹈的了解情况}}
    \begin{tabular}{ccc}
    \hline
    民间舞蹈艺术    & 了解人数  & 比例 \\
    \hline
    解表    & 4     & 1.38\% \\
    麻雀蹦   & 16    & 5.56\% \\
    花香鼓   & 24    & 8.33\% \\
    跳当当   & 8     & 2.77\% \\
    跳五猖   & 12    & 4.15\% \\
    打社火   & 28    & 9.71\% \\
    骆山大龙  & 20    & 6.94\% \\
    栖霞龙舞  & 63    & 22.19\% \\
    江浦手狮  & 20    & 6.94\% \\
    东坝大马灯 & 32    & 11.10\% \\
    都不了解  & 159   & 55.59\% \\
    \hline
    \end{tabular}%
\end{table}%

\begin{table}[htbp]
  \centering
  \caption{\bf{市民对南京市传统手工技艺的了解情况}}
    \begin{tabular}{ccc}
    \hline
    传统手工技艺    & 人数    & 比例 \\
    \hline
    云锦    & 136   & 47.49\% \\
    南京剪纸  & 80    & 28.03\% \\
    金陵金箔  & 33    & 11.44\% \\
    金陵折扇  & 74    & 25.74\% \\
    仿古牙雕  & 10    & 3.43\% \\
    南京木雕  & 31    & 10.87\% \\
    金陵竹刻  & 21    & 7.44\% \\
    雕花天鹅绒 & 5     & 1.71\% \\
    都不了解  & 106   & 37.06\% \\
    \hline
    \end{tabular}%
\end{table}%

\begin{table}[htbp]
  \centering
  \caption{\bf{市民对南京市秦淮八绝的了解情况}}
    \begin{tabular}{ccc}
    \hline
    秦淮八绝了解程度 & 赞成人数    & 比例 \\
    \hline
    全部吃过  & 7     & 2.29\% \\
    吃过部分  & 67    & 23.43\% \\
    全部听说过 & 5     & 1.71\% \\
    听说过部分 & 80    & 28.00\% \\
    不了解   & 127   & 44.57\% \\
    \hline
    \end{tabular}%
\end{table}%

从表5可以看出,大多数受访者对南京市的传统曲艺文化并不了解,其数量达到了近6成;而对南京单项传统曲艺文化了解的受访者,人数大多不超过两成,最少的不到一成。对于这部分人群,我们再根据其了解的传统曲艺文化分布情况进行分析,其结果如表9所示。从表9 可以看到,南京白局作为国家非物质遗产代表项目,了解它的人数相较之下较多,与南京白话持平,约占四成;评话在了解人数其次;阳腔目连戏与洪山戏的了解人数较少,分别为12.12\%与15.13\%。

类似的情况也出现在了表6当中。受访者对南京市民间舞蹈也是不了解的居多,其数量超过5成;除栖霞龙舞有2成左右的人了解之外,其余各项的了解人数均在1成左右或不到1成。我们也按照上面的做法,制成了表10。从中可以看出,受访者对南京市民间舞蹈的了解状况不是非常的乐观。作为国家级非物质遗产代表项目的驼山大龙和东坝大马灯,其了解人数不到1/4。 而了解人数最少的解表,只有3\% 的人知道。

传统手工技艺的情况则相对而言比较乐观,从表7可以看出,作为联合国非物质遗产代表名录之一的南京云锦制造技术的了解人数达到了约5 成,甚至超过了都不了解的人数(约4成)。但同样在联合国非物质遗产代表名录的南京剪纸就显得稍微落寞了,与金陵折扇的了解人数大致相等,约为3成。而其他传统手工技艺的民众了解情况则不是很好,其中最少的雕花天鹅绒,只有1\%的人了解。从表11中不难看出,民众对南京市传统手工技艺的了解集中在南京云锦、剪纸及折扇,其他的项目较少问津。

对于有前国家副主席荣毅仁“小吃好吃”题词加持的秦淮八绝而言,表8显示,也仍然有超4成的受访者表示并不了解;全部吃过或全部听说过的受访者合占不到5\%。

综上可以看出,对于南京市非物质文化遗产及相关民间传统文化,除极具地域特色的少数项目外,民众的认知度并不高,传统文化在现代显示出薄弱的生命力。

\begin{table}[htbp]
  \centering
  \caption{\bf{市民对南京市传统曲艺的了解分布情况}}
    \begin{tabular}{ccc}
    \hline
    传统曲艺文化  & 了解人数  & 比例 \\
    \hline
    白局    & 52    & 42.44\% \\
    白话    & 52    & 42.44\% \\
    评话    & 44    & 36.37\% \\
    阳腔目连戏 & 15    & 12.12\% \\
    洪山戏   & 18    & 15.13\% \\
    \hline
    \end{tabular}%
\end{table}%

\begin{table}[htbp]
  \centering
  \caption{\bf{市民对南京市民间舞蹈的了解分布情况}}
    \begin{tabular}{ccc}
    \hline
    民间舞蹈艺术 & 了解人数  & 比例 \\
    \hline
    解表    & 4     & 3.12\% \\
    麻雀蹦   & 16    & 12.52\% \\
    花香鼓   & 24    & 18.75\% \\
    跳当当   & 8     & 6.23\% \\
    跳五猖   & 12    & 9.35\% \\
    打社火   & 28    & 21.87\% \\
    骆山大龙  & 20    & 15.64\% \\
    栖霞龙舞  & 63    & 49.97\% \\
    江浦手狮  & 20    & 15.64\% \\
    东坝大马灯 & 32    & 24.99\% \\
    \hline
    \end{tabular}%
\end{table}%

\begin{table}[htbp]
  \centering
  \caption{\bf{市民对南京市传统手工技艺的了解分布情况}}
    \begin{tabular}{ccc}
    \hline
    传统手工技艺 & 了解人数  & 比例 \\
    \hline
    云锦    & 136   & 75.45\% \\
    南京剪纸  & 80    & 44.54\% \\
    金陵金箔  & 33    & 18.18\% \\
    金陵折扇  & 74    & 40.90\% \\
    仿古牙雕  & 10    & 5.46\% \\
    南京木雕  & 31    & 17.28\% \\
    金陵竹刻  & 21    & 11.82\% \\
    雕花天鹅绒 & 5     & 2.72\% \\
    \hline
    \end{tabular}%
\end{table}%



\subsection{南京市“非遗”发展前景及相关情况的民间性看法}

在民众认知度不高的情况下,对于南京市非遗及相关传统的文化的认同感、自豪感认知情况又会如何。我们对南京市非物质文化遗产及相关民间传统文化的发展前景、现存问题及发展建议作了民间性的看法调查,其结果如表12-14所示。

\begin{table}[htbp]
  \centering
  \caption{\bf{市民对南京市“非遗”及相关民间传统文化发展前景的看法}}
    \begin{tabular}{ccc}
    \hline
    发展前景  & 赞成人数    & 比例 \\
    \hline
    非常乐观  & 16    & 5.71\% \\
    比较乐观  & 186   & 65.14\% \\
    不乐观   & 75    & 26.29\% \\
    完全不看好 & 8     & 2.86\% \\
    \hline
    \end{tabular}%
\end{table}%



\begin{table}[htbp]
  \centering
  \caption{\bf{市民对南京市“非遗”及相关民间传统文化发展现存问题的看法}}
    \begin{tabular}{ccc}
    \hline
    现存的问题 & 赞成人数    & 比例 \\
    \hline
    外来文化和现代文化的冲击 & 188   & 65.71\% \\
    缺乏资金  & 62    & 21.71\% \\
    缺乏保护意识 & 150   & 52.57\% \\
    缺乏有效的保护机制 & 149   & 52.00\% \\
    缺乏传承人 & 176   & 61.71\% \\
    其他    & 25    & 8.57\% \\
    \hline
    \end{tabular}%
\end{table}%



\begin{table}[htbp]
  \centering
  \caption{\bf{市民对南京市“非遗”及相关民间传统文化发展最赞成的宣传方式}}
    \begin{tabular}{ccc}
    \hline
    宣传方式 & 人数    & 比例 \\
    \hline
    拍成纪录片,在电视台播放 & 123   & 42.86\% \\
    举办各种演出活动 & 44    & 15.43\% \\
    发放各种资料,如宣传册、海报等 & 16    & 5.71\% \\
    进校园活动,如演出、讲座等 & 87    & 30.29\% \\
    其他    & 16    & 5.71\% \\
    \hline
    \end{tabular}%
\end{table}%

令我们感到欣喜的是,即便民众对南京市“非遗”与民间传统文化情况的了解程度不高,但仍表示了对其文化发展前景的积极看法。受访者中持偏乐观看法的人数超过7成,仅有不到3\%的受访者完全不看好南京市“非遗”与民间传统文化的发展。

而在回答南京市“非遗”与民间传统文化继承与发展中存在的问题时,民众认为外来文化和现代文化的冲击,缺乏传承人,缺乏保护意识,缺乏有效的保护机制是传统文化发展面临的主要问题。同时,也有受访者指出,没有足够的宣传,仅仅依靠传统文化本身的魅力,是使得传统文化局限在本地,难以被其他地区了解的原因。亦有一小部分人认为南京市“非遗”与民间传统文化没有足够强的吸引力,缺乏鲜明特色是其认知度不高的原因。

针对这些面临的问题,我们不难看出:正确,或者说合适的宣传方式成为了民间文化能否在现代焕发鲜活的生命力的关键。受访者们最赞成的宣传方式有拍成纪录片在电视台播放,以及走进校园,通过演出、讲座等活动来进行宣传。也有受访者指出,可以联合扬州,无锡,苏州等其他非物质文化遗产资源比较丰富的地区,形成区域性的民间文化宣传结构,也可以是比较好的宣传方式。

\section{结论分析与建议}

\subsection{调查结果浅析}

如上文所言,民众对于南京市非物质文化遗产及相关民间传统文化的认知度不高,却有较好的认同感与自豪感。借助问卷调查,我们大致将原因分成以下几个方面:
\begin{enumerate}
  \item \textbf{方言遗失状况下带来的欣赏能力冲击。}

  随着南京市能说本地方言的人口数量逐渐降低,一些需要本地方言才能够得以理解的传统曲艺、语言文化变得难以被新南京人理解,毋论其他地区市民。在缺乏方言基础的情况下,民众对这些非物质文化遗产及相关民间传统文化的欣赏能力不够,进而失去关注度。
  \item \textbf{外来文化和现代文化的冲击下,传统文化略显无力。}

  以美国文化为代表的西方文化和现代文化给传统文化带来了一定的冲击。在这种冲击下,一方面人们对于传统文化的理解畸形,过分强调“与时俱进”而忽略了传统文化有其时代属性,过分要求传统文化为潮流“转型”。另一方面,一些传统文化由于内容和形式的陈旧,与时代脱节,发展环境艰难。传统文化陷入两难的境地。

  \item \textbf{缺乏传承人与有效的保护机制。}

  由于保护措施不当,许多非遗及传统文化缺少继承人,或缺少年轻的继承人。部分非遗的传承已出现问题,更勿论传播与发扬。而对于另一部分继承下来的非遗及传统文化,又面临着机制的保护不当,出现了非遗文化质量不高的缺点,无法在现代社会竞争。

  \item \textbf{缺乏足够的宣传,任凭传统文化“自生自灭”。}

  现代社会以不同于以往,单纯的依靠口碑“自生自灭”的继承、经营与发扬方式已不再适用。缺乏足够的宣传,使得民众根本没有渠道了解非遗及传统文化。对于非南京地区的市民而言,这一问题尤为突出。

  \item \textbf{民众对传统文化有一定的自信。}

  虽然在以上问题的影响下,传统文化在现代的生命力并不十分鲜活,但民众仍然对其保持了较高的自信与认同,对其发展前景有着较为乐观的评估。同时,也愿意参与到优秀传统文化的继承与发扬中来。

\end{enumerate}



\subsection{意见与建议}

针对上文中所提出的各项问题,我们给出如下的参考意见与建议:
\begin{enumerate}
  \item \textbf{加大对方言的保护力度,对于非遗及传统文化的欣赏给予一定指导。}

  对于南京市本地方言,需要加大保护力度,让方言走进生活而不是“记录”。可以适当发挥大众传媒及政府基础公共资源的作用,推行汉语普通话的基础上,进行地铁公交系统双语播报、部分本地方言播出、方言教育等等途径。对于一些非遗及传统文化,可以适当的给予民众在欣赏上的提示与指导,方便民众理解非遗与传统文化。

  \item \textbf{取其精华,去其糟粕;推陈出新,革故鼎新。}

  对于传统文化,取其精华,去其糟粕,不过分要求其与时代完全接轨,也不盲目照搬其中的陈旧部分。推陈出新,革故鼎新,在改革的过程中,要注意保留其根本属性,做到“不变质”。拒绝闭关主义、保守主义的同时,也拒绝历史虚无主义与民族虚无主义。

  \item \textbf{建立正确的保护机制,保障传承人的正常生活状况与传统文化的质量。}

  在现今环境下,传承人仅仅依靠非遗,往往难以获得较好的物质生活水平,因此,对传承人正常生活的保障,也成了非物质文化遗产继承和发扬过程当中非常重要的一环。同时,也要提高非遗及传统文化传承人的文化传承意愿,以及在进行相关传承时的必要保证,做到“有物可传”。还要对传承人的文化传承质量进行监督,避免“越传越差”的情况。

  \item \textbf{加大宣传力度,发挥区域优势。}

  通过制成纪录片,在电视台播放;发放各种资料;进入校园,进行演出讲座等活动的形式进行传统文化的宣传。也可以联合其它周边地区,形成区域性的民间文化宣传结构,发挥地域优势。
\end{enumerate}

\appendix
\section{参考文献}
\begin{enumerate} [1 ]
  \item  \url{http://www.ihchina.cn/3/10356.html}
  \item  \url{http://www.ihchina.cn/3/10357.html}
  \item  \url{http://www.ihchina.cn/3/18945.html}
  \item  \url{http://www.ihchina.cn/3/10315.html}
  \item  \url{http://www.ihchina.cn/3/10316.html}
  \item  \url{http://www.gov.cn/flfg/2011-02/25/content_1857449.htm}
  \item  石冰倩,廖若琳,郭宁静,王荃,姜颖慧.非物质文化遗产在“互联网+”模式下的传承与创新——以南京云锦为例[J].中小企业管理与科技(上旬刊),2018(01):75-76.
  \item  付启元.南京非物质文化遗产保护与利用研究[J].南京晓庄学院学报,2008(02):48-55.
  \item  陈燕,喻学才.关于南京非物质文化遗产的保护与经营刍议[J].东南大学学报(哲学社会科学版),2006(S2):107-109.
  \item  刘舟. 南京金箔工艺研究与技艺传承的再思考[D].山东艺术学院,2016.
  \item  陆亚华,游小敏,狄俊,张发勇.非物质文化遗产英文网络外宣的问题与对策——以青奥背景下的南京为例[J].学理论,2013(32):201-202.
  \item  王春雩.地方性非物质文化遗产传承困境与对策——以南京白局为例[J].现代经济信息,2015(05):430.
\end{enumerate}



\section{附录}

\subsection{分工及感想}

\subsubsection{许晓明}

分工:

后期资料整理、论文主笔。

感想:

最初拿到问卷情况的时候我其实是很困惑的,尤其是在整理完所有的数据之后,这种感觉更加强烈了。传统文化是一个很大的方面,即便缩小到地域的范围也还是非常大。从这样一个角度进行切入,很有可能会写得比较空泛。而这恰恰是我在写论文的时候比较忌讳的一点。论文的写作应该要有意义,通过一篇论文,应该要给读者传达出一个有现实作用的东西。而同时这个出发点又很有可能落入另一个极端,那就是论文的写作缺乏了“自知之明”,因为传统文化是一个老生常谈的问题,有很多专门从事这方面研究的专家学者,也在进行相关的整理和策略考量,不可能说像我们大学生这样的群体,仅仅通过一个社会实践就可以去解决这样一个问题,这是不现实的。因此,怎样选择一个角度进行切入,就成了关键性的问题。在查阅相关文献的过程当中,同时参考最终调查问卷结果的情况下,想到从非物质文化遗产进行切入,最终落脚在传统文化的现代生命力上。为了更好的完善论文的历史背景,查阅了很多关于非物质文化遗产的资料,包括像联合国教科文组织的一些文件及国家的一系列法规文件,也确定下了文化回归民众的主题。这也是我全篇不断地强调传统文化应该回归到民间,回归到生活,而不是进入博物馆的原因。当然,实际操作过程当中,可能和我自己的设想并不完全一致,包括像最后《意见与建议》的部分,就缺乏了大学生做社会实践所应展现的特色而略显官方、流于俗套了。

同时,这也是我第一次使用 \LaTeX 写论文,这个想法很早前就有,也希望借此可以更加扎实自己关于 \LaTeX 的一些功底,当然期间也遇到很多关于软件方面的一些问题,其中大部分还是克服了,虽然也有一些不足,像本来有设想加入一些图片的元素在里面,但是因为一些操作上的问题,最终只好作罢。最终论文成品还算让我满意,因而也比较庆幸。

总而言之,这是我少数几次写的比较大,比较正式的论文。也是我第一次以文化作为大的主题来写论文,在写作的过程当中,也确实感受到传统文化在现代所焕发出来的生机,可能并不是那么鲜明。这从本质上而言并不是一个问题,更确切的说应该是一个现象。而我们所思考的就是这个现象是否是正确的,以及在这个现象背后,我们应该做的是什么。


\subsubsection{袁善帅}
分工:

写问卷、传播问卷。

感想:

在本次社会调查活动中,问卷的初稿是由我来写的。最初选的主题和南京文化相关,所以我就上网找了找有关南京文化的资料。找到的资料很丰富,涉及到南京特色的文学、书画、音乐、语言、曲艺等,但是内容比较生僻,至少我这个不是南京本地人的大学生很多都不了解。我们的问卷由单选和多选组成,多选题中会有一些题,选项中列出南京特色文化的名称,让被访者勾选了解过的,然而调查后得到的结果却是超过半数的人对选项中的特色文化一无所知。由此我反思,我在设计问卷的时候应当充分考虑到被访者的知识结构,不能选择过于生僻的内容进行调查,如果大部分人对我所要调查的内容都不熟悉,那我在问卷最后询问他们的意见和建议也就没有多大意义了。

不过大家对调查内容的陌生也从另一个方面反应了南京市传统文化继承和发展的状况不容乐观。许多南京的传统文化如特色的语言曲艺等因为和现代生活脱节,只有少数爱好者掌握其技艺,再加之宣传力度不够,很多本地人都意识不到这些传统技艺的存在,更别提继承和发展了。所以我认为对待传统文化不能任其自生自灭,要积极的去宣传;另外,对于文化本身也要积极的去发展,尝试将传统文化融入到人们的日常生活中。传统文化不应该被束之高阁,它应当是生活的一部分,使人们的生活更加有深度。

这次的受访者以大学生群体为主,虽然多数受访者对南京传统文化的发展前景持较为积极的态度,但我们大学生做的还远远不够。我们要行动起来,做传统文化的宣传者、继承者和发展者。继承和发扬中华民族的传统文化,从宣传本土文化开始做起。



\subsubsection{周宏伟}
分工:

传播问卷。

感想:

我们加入到南理工的大家庭,与南京结下了不解之缘。因此,我们的社会调查选择的主题是南京的传统文化。南京白局,金陵竹雕,金陵葫芦彩绘,南京剪纸等等都是优秀的非物质文化遗产,而如今它们却被现代化的浪潮冲得七零八落。作为六朝古都的南京,自古以来就是人杰地灵之地,它的传统文化自然是相当出彩,引人入胜的,然而如今却只剩一些年长的南京本地人才有所耳闻,实在是可惜,可叹。我们作为新世纪的大学生,时代的弄潮儿,应当肩负起保护并发扬传统文化的责任。南京的传统文化,以至于中华民族的传统文化,都是应当且值得传承的。公众应当由点到线,由线到面地普及传统文化,让社会中的每一名角色都去理解,去保护,去发扬那已经被掩埋在角落的珍贵的传统文化。

\subsubsection{邹嘉璇}

分工:

资料收集、传播问卷。

感想:

‌此次社会调查,我们以南京市为例从非物质文化遗产及相关民间传统文化传承及传播情况看传统文化的生命力,通过发放问卷的方式进行了社会调查。从调查结果可以看出,我们学生即使在南京生活学习还是对南京的传统曲艺和民间舞蹈不够了解,即使是南京本地人对当地的曲艺还是不够了解。通过此次调查我觉得人们对南京当地的传统文化不够了解,希望相关文化的传统人能通过拍纪录片等方式向世人展示南京传统文化的魅力,让这些传统文化能够更加灿烂地传承与发扬下去。

\subsection{问卷式样}

\begin{center}
  \large 关于南京传统文化继承发展状况的调查问卷
\end{center}
\small

    您好!我们是南京理工大学调研小组的学生。为了了解南京传统文化继承发展状况,我们在网络上开展这项调查。本次问卷大约花费您10分钟的时间。\textbf{ 我们的研究只代表学术立场,不代表任何倾向,答案也没有正确错误之分,请根据您的实际情况填写。}
    另外,我们将会对您的答案予以保密,请您放心!

     感谢您的支持与合作!
     \begin{enumerate}
       \item 您的性别:\emph{单选题}
       \begin{itemize}
         \item 男
         \item 女
       \end{itemize}
       \item 您的年龄: \emph{单选题}
       \begin{itemize}
         \item 18岁以下
         \item 18-25岁
         \item 25-35岁
         \item 35岁以上
       \end{itemize}
       \item 您是否是南京人:(此问选择“不是”则不用回答下一题) \emph{单选题}
       \begin{itemize}
         \item 是
         \item 不是
       \end{itemize}
       \item 您是否会说南京话: \emph{单选题}
       \begin{itemize}
         \item 会
         \item 不会
       \end{itemize}
       \item 以下南京的曲艺文化您了解的有哪些: \emph{多选题}
       \begin{description}
         \item{-} 白局
         \item{-} 白话
         \item{-} 评话
         \item{-} 阳腔目连戏
         \item{-} 洪山戏
         \item{-} 都不了解
       \end{description}
       \item 以下南京的舞蹈您了解的有哪些: \emph{多选题}
       \begin{description}
         \item{-} 解表
         \item{-} 麻雀蹦
         \item{-} 花香鼓
         \item{-} 跳当当
         \item{-} 跳五猖
         \item{-} 打社火
         \item{-} 骆山大龙
         \item{-} 栖霞龙舞
         \item{-} 江浦手狮
         \item{-} 东坝大马灯
         \item{-} 都不了解
       \end{description}
       \item 以下南京的工艺您了解的有哪些: \emph{多选题}
       \begin{description}
         \item{-} 云锦
         \item{-} 南京剪纸
         \item{-} 金陵金箔
         \item{-} 金陵折扇
         \item{-} 仿古牙雕
         \item{-} 南京木雕
         \item{-} 金陵竹刻
         \item{-} 雕花天鹅绒
         \item{-} 都不了解
       \end{description}
       \item 您对南京美食中的秦淮八绝了解多少: \emph{单选题}
       \begin{itemize}
         \item 不了解
         \item 听说过部分
         \item 全部听说过
         \item 吃过部分
         \item 全部吃过
       \end{itemize}
       \item 您认为目前南京传统文化的发展前景如何: \emph{单选题}
       \begin{itemize}
         \item 完全不看好
         \item 不乐观
         \item 比较乐观
         \item 非常乐观
       \end{itemize}
       \item 您认为目前南京传统文化的的发展和传承存在哪些问题: \emph{多选题}
       \begin{description}
         \item{-} 外来文化和现代文化的冲击
         \item{-} 缺乏资金
         \item{-} 缺乏传承人
         \item{-} 缺乏保护意识
         \item{-} 缺乏有效的保护机制
         \item{-} 其他\_\_\_\_
       \end{description}
       \item 在宣传方面,您最赞成以下哪种方式宣传南京传统文化: \emph{单选题}
       \begin{itemize}
         \item 拍成纪录片,在电视台播放
         \item 举办各种演出活动
         \item 发放各种资料,如宣传册、海报等
         \item 进校园活动,如演出、讲座等
         \item 其他\_\_\_\_\_
       \end{itemize}
     \end{enumerate}

\subsection{组员联系方式}
  \begin{enumerate}[(i).]
    \item 许晓明 18851196872
    \item 袁善帅 13770928108
    \item 周宏伟 18260081020
    \item 邹嘉璇 18260036267
  \end{enumerate}
\end{document}

\end{document}
