\documentclass[12pt]{article}% 这是主要的格式。

\usepackage[UTF8]{ctex}
\usepackage{appendix}
\usepackage{enumerate}
\usepackage{amsmath}
\usepackage{graphicx}
\usepackage{cite}
\usepackage{geometry}
\usepackage[section]{placeins}
\usepackage[colorlinks,linkcolor=blue]{hyperref}
\title{一生只做好一件事}
\author{“学习王泽山院士”——《大家:2017年度国家最高科学奖获得者》观后感 \\9161040G0734许晓明 }
%\author{}
\date{}
\geometry{scale=0.75}

\begin{document}%文档从这里开始。

\numberwithin{footnote}{section}
\renewcommand{\contentsname}{\centering 目录}
\renewcommand{\tablename}{表}
\renewcommand{\figurename}{图}
\renewcommand\refname{参考文献}
\renewcommand\appendix{\setcounter{secnumdepth}{0}}
\renewcommand\abstractname{摘要}



\maketitle
%\abstract{非物质文化遗产增强了人们对文化多样性和人类创造力的尊重。而2017年3月1日实施的《南京市非物质文化遗产保护条例》则正式使南京市非物质文化遗产保护工作进入了新阶段。但非物质文化遗产及相关民间传统文化在传承及传播过程中仍然存在一些问题,民众对非遗及相关民间传统文化的认知度还不是很高。伴随着民众对非遗及相关民间传统文化自豪感与认同感的提高,我们可以从中窥探传统文化的现代生命力。本文通过问卷调查的形式,分析了南京市非物质文化遗产及相关民间传统文化传承和发展上的困境,总结了造成这些困境的原因并给出了相应的建议。}
%\\
%\textbf{关键词}:非物质文化遗产\ 民间\ 传统文化\ 传承\ 传播\ 南京市

\large\tableofcontents
\newpage
\large 2015年9月3日,世界反法西斯战争胜利70周年的阅兵仪式上,多种中大口径火炮武器作为我国最先进的火力系统,驶过天安门广场。在感叹我国国防力量强大的同时,却很少有人知道,这些火炮的动力系统-火炸药-的背后,传承着王泽山和团队的汗水与信念。为了将我国的火炸药技术,提升到世界先进水平,他付出了60余年,并将继续在人们看不见的地方,发光发热。
\appendix
\section{引言}
火炸药的历史可追溯到我国古代四大发明之一的黑火药。虽然黑火药被世界公认为是现代火炸药的鼻祖,但在世界近代几百年的时间里,中国的火炸药技术却一直大幅落后于西方。

新中国成立后,我国逐渐建立起了自己的火炸药研制体系,火炸药技术的发展水平也在经历了几次波折之后,逐步提高,并在世界火炸药领域拥有了属于自己的一席之地。

而完成从初入邻域到拔得头筹的飞跃性进展,则可以说主要归功于王泽山院士及他的团队。作为我国火炸药领域的领军人物,王泽山获得两次国家技术发明一等奖和一次国家科技进步一等奖,成为我国唯一一位三次获得国家科技一等奖的科学家。但因为保密的工作性质,他很少出现在媒体的报道中。
\section{王泽山院士成就简述}
如今我们对王泽山院士成就的认知往往停留在火炮等军事邻域,却未曾注意到,他也是火炸药民用价值开发的第一人。

火炸药在一个国家军事实力发展的过程当中,扮演着不可替代的重要角色。火炸药的轮储是每个国家必须重视的国防战略需要。正因如此,我国每年都会有上万吨退役的火炸药产生。由于火炸药有毒,易燃易爆,退役火炸药对我国社会安全和环境构成了严重的威胁,传统的处理方法是露天焚烧,但这样的方法既存在安全隐患,也污染环境,浪费资源。

为了解决这些问题,上世纪80年代,王泽山针对不同类型的火炸药,提出了资源化利用的技术途径。通过一系列物理和化学方法,不仅降低了安全风险和环境污染,还因为将废弃的军用火炸药改为民用节约的消费费用,并创造了非常可观的经济效益。

除了民用,火炸药在国防邻域的应用也很重要。王泽山带领团队解决的领先世界的低温度感度技术,成为其中的典范。对于这个技术,欧美几个发达国家曾联合研制多年,但并没有取得理想的结果。王泽山和团队不断尝试,他们打破原有规律,构建的火药燃速和燃面的等效关系,并发现了能够弥补温度影响的新材料。如今王泽山和团队发明的这些火炸药,早已装备于我国的主战火炮和坦克,作为我国国防力量的重要组成,这些武器的性能已经摆脱了环境温度的影响,而国外的低温度感度火药技术,至今仍没有太大进展。

在退休之后的第22年,王泽山和团队利用他们最新的发明技术,将我国火炮射程提升了20\%,这在世界范围内都是一个巨大的突破,王泽山也因此获得了第三个国家科技一等奖。
\section{一生只做好一件事}
一生只做一件事,对王泽山院士而言这件事无疑就是对火炸药技术的研究。他潜心研究火炸药,默默耕耘在国防事业前沿。他"一生只做好一件事"的精神,成为最宝贵的精神财富。

我们往往看到了王泽山院士取得的成就,却忽略了他在我们看不见的地方辛勤耕耘,默默付出。厚积薄发,方一鸣惊人。为了能够尽快研制出等模块火药,退休后的20年,王泽山一刻也不曾停歇,哪有问题需要解决,他就会出现在哪,一年的时间,他有半年都在出差。对于这样的王泽山院士,我们敬佩之余,却也多了一份心痛。

直到现在,每到靶场实验室,80多岁的王泽山一如年轻时那样,依旧亲自参与,冬天到最冷的地方,夏天去最热的地方,目的就是能够获得最详尽的数据和结果。

越是艰苦的地方越冷清。江山代有才人出,各领风骚数百年。科研与国防需要阅历,但却不应当是老教授们的孤军奋战;年轻的血液注入其中,方迎来一场万古长青。
\section{参考资料}
\begin{enumerate}[1 ]
\item \url{http://tv.cctv.com/2017/05/11/VIDEl7ngYqAt5FeN2R6RKWtq170511.shtml}
  \item 张晔.王泽山:六十年书写火炸药传奇[J].智慧中国,2018(01):42-45.
  \item 张晔. 王泽山:以身相许火炸药[N]. 科技日报,2018-01-09(006).
  \item 周烨.王泽山:让中国火炸药技术傲视全球[J].中国科技产业,2017(05):36-37.
  \item 南黎.成功源于超越性追求——记南京理工大学王泽山院士[J].国防科技工业,2006(09):44-45.
  \item 苏言. 勇当科技创新“领跑者”[N]. 新华日报,2018-02-02(001).
  \item 谈洁,杨萍.六十年磨一剑 为“战争之神”攻克世界难题[J].科学大众(中学生),2017(03):7-8.
\end{enumerate}

\end{document}
