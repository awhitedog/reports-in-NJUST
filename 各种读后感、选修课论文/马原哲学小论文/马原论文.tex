\documentclass[12pt]{article}% 这是主要的格式。

\usepackage[UTF8]{ctex}
\usepackage{appendix}
\usepackage{enumerate}
\usepackage{amsmath}
\usepackage{graphicx}
\usepackage{cite}
\usepackage{geometry}
\usepackage[section]{placeins}
\usepackage[colorlinks,linkcolor=blue]{hyperref}
\title{从唯物辩证法的角度看国外影视作品成功背后的原因及启示——以日剧《unnatural》为例}
\author{0G07班 9161040G0734 许晓明 41号}
%\author{}
\date{}
\geometry{left=2.0cm,right=2.0cm}

\begin{document}%文档从这里开始。

\numberwithin{footnote}{section}
\renewcommand{\contentsname}{\centering 目录}
\renewcommand{\tablename}{表}
\renewcommand{\figurename}{图}
\renewcommand\refname{参考文献}
\renewcommand\appendix{\setcounter{secnumdepth}{0}}
\renewcommand\abstractname{摘要}



\maketitle

\abstract{随着经济的发展,人们的生活水平不断提高,日益增长的文化需求对影视作品的质量提出了一定要求。但目前中国影视作品的现状不容乐观,与此同时,与我们隔海相望的日本,却无论是影视作品的发展时间还是发展阶段上,都一定程度上领先于国内。其中,2018年日剧《unnatural》更是带来了现象级的影响。4月,《unnatural》引进国内,也掀起国内的热潮。这样一部仅有10集的日剧,为何会有如此强大的魅力?本文简单分析了其成功背后蕴含的哲学道理,指出其大热的背后并非是简单的偶然,并探讨其成功因素能否为国产影视作品的发展提供相应的借鉴依据。}
\\ 

\textbf{关键词}:影视作品\ 唯物辩证法\ 唯物论\ 原因分析 
\newpage
\tableofcontents
\newpage
\section{导言}

\subsection{高质量影视作品的需求与当前影视作品现状}
随着经济的发展,人们的生活水平不断提高,日益增长的文化需求对影视作品的质量提出了一定要求。优秀的影视作品在输出文化的同时,能正确引导观众的价值取向,弘扬社会正能量,成为社会和谐稳定发展的精神助推剂。日前,国家广播电视总局宣传司司长高长力发表讲话,表示今后广播电视节目必须继续遵循 “小成本、大情怀、正能量”的自主创新原则,不讲排场、比阔气、拼明星。这对影视作品的发展提出了新要求。

但目前中国影视作品的现状却是泥沙俱下:许多质量不合格、滥竽充数的作品大量涌现在观众眼中。以《锦绣未央》、《极光之恋》、《孤芳不自赏》等等一大批影视作品为代表,国产影视作品似乎陷入了纯粹收益至上、抄袭类IP盛行、演技薄弱“小鲜肉”当道、剧本漏洞明显甚至消费情怀等等问题之中。这之中固然也有像《人民的名义》、《红色》等优秀作品的出现,但也终究是凤毛麟角、昙花一现,国产影视作品似乎陷入微妙的探索阶段。

虽然社会各界对与好的影视作品的定义不尽相同,但大体上有着一定的共识:一个好的影视作品,应当是雅俗共赏的,即有拥有一定的艺术价值,同时又是广大群众喜闻乐见、愿意接受的。优秀的影视作品,不仅对行业由推动作用,更对社会有价值贡献。


\subsection{优秀国外影视作品案例《unnatural》}
而与我们隔海相望的日本,无论是影视作品的发展时间还是发展阶段,都一定程度上领先于国内。2018年1月,一部仅10集的电视剧《unnatural》(中译《非自然死亡》,下简称《不自然》)一经播出,便引起极大反响。创造日本平均收视率	11.06\%,最高收视率13.3\%的成绩。不光是收视率,在口碑、经济效益、社会话题度上,《不自然》都获得了良好的效果。

随着2018年4月,优酷、哔哩哔哩等视频网站将其引进,《不自然》又在国内掀起一阵热潮,豆瓣评分为9.1,《南方都市报》评论其“紧凑的叙事和折射出的社会现状,让该剧成为了一部有吸引力的剧集。”、“借法医的职业视角,折射在破案过程出现的种种社会现状”,《新京报》评论该剧“对于“非自然死亡”背后真相的追求,反映出诸多人性、生死观、社会矛盾等命题“。

《不自然》的主要故事叙述很简单,讲述在“非自然死亡原因研究所”(简称“UDI”)工作的三澄美琴是专门探查死者死因的解剖医生。她最不能容忍的是对“非自然死亡”不闻不问。在她看来,“非自然”的背后必定有着需要法医来追究的真相,比如伪装杀人、医疗失误、未知的疑难疾病等等。然而,在日本,很多非自然死亡的死者都未经解剖就火化了。美琴与她那些个性鲜明的同事们一起,向这样的现实发起了挑战。

然而就是这样的故事,却给2个国家的许多观众都带来了不同却又绝佳的观赏体验。
\subsection{问题的提出}

《不自然》在日本、中国均取得了巨大成功,其带来的社会话题度甚至一度成为现象级的影响。那么,《不自然》的成功背后,我们可以看出哪些哲学原理呢?这些因素又能否为我国影视作品所用,以期快速走出探索阶段呢?为此,笔者结合唯物辩证法,进行了简单的分析。


\section{《unnatural》成功背后的原因分析}
《不自然》作为影视作品,成功背后并非偶然,而是多方面的因素,在这一部分,笔者按照演员、剧情、细节等内容,结合唯物辩证法浅析其成功的原因。
\subsection{演员(人物)}
唯物辩证法认为,矛盾存在于一切事物中,并且贯穿于每一事物的始终,即事事有矛盾,时时有矛盾。其中,事物有主要矛盾和次要矛盾之分,主要矛盾是指在事物发展过程中处于支配地位,对事物发展起决定作用的矛盾,而次要矛盾则处于从属地位,不起决定作用。而矛盾又有主要方面和次要方面之分,主要方面在事物内部居于支配地位,起主导作用,而次要方面则是处于被支配地位,不起主导作用。主要矛盾的主要方面决定着事物的性质与发展方向。

对于影视作品而言,最先为观众所知的演员(人物)无疑成为了主要矛盾的主要方面。《不自然》能取得较大成功,各位主要演员功不可没。主演娴熟的演技,使得人物的表现非常亮眼,人物角色性格鲜明,各有特色,都非常让人喜爱。

主角们“接地气“,是一群不完美的职场人,却因为工作的高度专注和积极进取精神而让角色闪闪发光。各个角色突破了个人英雄主义的常规套路,化为一个个普通的“人”,正是这些人物角色中的各种不完美才让这部职场剧更真实贴近现实生活,让这些角色塑造得栩栩如生,更加惹人喜爱。
\subsection{剧情}
唯物辩证法的联系观点要求正确理解整体与部分的关系。整体是处于统帅的决定地位的,整体的性能状态及其变化会影响到部分的性能状态及其变化;而整体又是由部分构成的,因此部分会制约整体,甚至在一定条件下,关键部分的性能状态会对整体的性能状态起决定作用。

《不自然》的剧情则很好地做到了整体与部分的统一:从整体上看,全剧围绕主线展开,单集又分别讲述一个小故事,在每一个小故事之间,加以铺垫,使得10集的剧情紧凑而又一气呵成,充分做到合理分配剧情进度,达到最优的效果。从部分上看,单集借助剧情的反转避免拖沓,让观众快速跟随故事的发展,每一个小故事折射社会问题,以小见大,以剧情的深度实现对整体的把握起到画龙点睛的作用。整体与部分相统一,使得剧情连贯,没有太大的漏洞,极大的提高了观众的视觉体验。


\subsection{细节}
唯物辩证法强调量变和质变的辨证关系:事物的变化发展都是量变和质变的统一,一切事物的发展都是首先从量变开始的,量变是质变的前提;量变达到一定程度后引起质变,质变是量变的结果。事物在量变和质变的两种状态中发展的。

《不自然》对细节的把握则可以说准确的符合了量变产生质变的条件。剧中对于专业知识的考究、职业特点的把握到了近乎苛责的程度,正是这种细节上的认真,积累量变,实现剧集质量的质变。尤其是剧中的插曲lemon的使用,通过在十分恰当的位置插入歌曲,对气氛的渲染作用几乎达到高潮。在结尾集数,lemon的运用使得整剧在统一中达到质的升华。
\subsection{说教}
唯物辩证法强调联系的普遍性与客观性。事物处于普遍联系之中,没有任何一个事物是孤立存在的,整个世界就是一个普遍联系的统一整体。事物的联系又是客观的,不以人的意识为转移,人们不能否认和割断事物之间的客观联系,也不能主观臆造联系。


日剧最大的特点之一就是说教,《不自然》也不例外。但日剧的说教不等同于机械地道德灌输,更强调价值观的碰撞。表现在《不自然》中则是价值观与事物的联系,主角们作为法医的价值观与工作的联系,对生活事情的处理价值观与日常背景的联系等等。《不自然》充分考虑到联系的普遍性与客观性原理,尊重事物的联系,展现联系状况下呈现出的不同价值观,不强行进行价值判断,而是将判断的最终决定交给观众,自身则更注重文化的输出。
\section{《unnatural》的成功带来的启示}
既然《不自然》的成功并非偶然,那么其成功背后的因素,国产影视作品也可以进行借鉴,这里笔者给出如下的建议:
\begin{enumerate}[a).]
\item \textbf{重视主要矛盾的主要方面对事物发展的作用:演员的选取上更多注重实力,人物的塑造上更多贴近生活。}

演员是影视作品最先让观众接触的部分,在当今“流量明星”当道的情况下,对有实力的演员的选择,成为作品得到较好呈现的基础。在人物塑造上,过分强调“主角万能”反而会脱离群众,产生距离。
\item \textbf{正确处理好整体与部分的关系,达到系统的最优化:注重剧情的打磨,减少剧情硬伤。}

如果是演员是让观众获得良好印象的一步,那么剧情便是吸引观众留下的关键。紧凑而合理的剧情甚至能一定程度上弥补演员的不足。消除剧情上的硬伤,可以给观众带来更好的体验。
\item \textbf{把握量变与质变的关系,要重视量的积累、抓住时机促成质的飞跃:适当考虑细节,营造“走心”感觉。}

对于细节的把握可以让观众认识到制作者对于作品的重视程度。配乐、专业知识等内容出现问题会暴露制作费的不认真。粗糙的作品或许可以凭借演员和剧情来弥补,却很难让观众选择再回味。
\item \textbf{尊重联系的普遍性、客观性,认识和把握真实联系,具体分析事物之间的联系,根据固有联系改变影视作品的状态:注重文化输出,弱化灌输式教育}

影视作品探讨价值层面的取向,可以有一定的导向,但不宜有过于明确的“指示”,也不宜出现过多的、单一的价值判断。与此同时,价值的多样性、文化的合理性是影视作品从“合格”到“佳作”晋升中非常重要的一步。注重文化输出的影视作品不会才可能更深刻,而不是流于表面。
\end{enumerate}

\appendix
\section{参考资料}
\begin{enumerate} [1 ]
  \item  袁智忠. 新时期影视作品道德价值取向及其对青少年的影响研究[D].西南大学,2009.
  \item  \url{http://epaper.oeeee.com/epaper/C/html/2018-02/01/content_7787.htm#article}
  \item  \url{http://epaper.bjnews.com.cn/html/2018-02/09/content_711047.htm?div=-1}
  \item 张遥. 当代中国网络影视评论研究[D].吉林大学,2014.
  \item  沈吴莎. 当下我国影视IP改编热现象研究[D].湖北师范大学,2017.
  \item  黄小铭.影视作品对高校思想政治教育工作的影响研究[J].吉林师范大学学报(人文社会科学版),2011,39(04):109-112.
  \item  高雄杰.试论影视作品中物件细节的叙事功能[J].现代传播(中国传媒大学学报),2010(10):75-78.
  \item   \url{https://www.zhihu.com/question/265525677/answer/316963084}
\end{enumerate}



\section{论文原创性说明}
\textbf{论文为作者原创}


\end{document}
