\documentclass[12pt]{article}% 这是主要的格式。

\usepackage[UTF8]{ctex}
\usepackage{appendix}
\usepackage{enumerate}
\usepackage{amsmath}
\usepackage{graphicx}
\usepackage{cite}
\usepackage{geometry}
\usepackage[section]{placeins}
\usepackage[colorlinks,linkcolor=blue]{hyperref}
\title{英雄豪气,荣膺长存}
\author{——《林海雪原》(人民文学出版社2012年,曲波著)}
%\author{}
\date{9161040G0734 许晓明}
\geometry{scale=0.75}

\begin{document}%文档从这里开始。



\maketitle
提到“林海雪原”四字,本是便想到《智取威虎山》,想到杨子荣与座山雕。如今,读罢《林海雪原》,便更能回想一段波澜壮阔的历史。掩卷沉思,亦收获不少:

本以为杨子荣这个人物,当是书中最主要的人物,然而通过阅读才发现他仅仅是属于第二人物来叙写的,
本书的主要人物是描写年轻有为的军事将领少剑波,书中详细刻写了他沉着冷静、不骄不燥、英勇善战、足智多谋的成长历程以及简单过度的爱情故事,更为主要的可能就是作者本人的缩影。但毕竟,我是“认识”杨子荣多的,言语之间,难免还是误将他做了中心角色。

在冰天雪地的东北大平原,有一支东北人民解放军小分队,在团参谋长的率领下深入林海雪原执行剿匪任务。这股匪徒是原国民党的败兵流窜到我军后方的。就这样,一场斗智斗勇的故事展开了。

战士们同甘共苦,排除万难,忍受了常人难以忍受的艰苦、克服了难以想像的困难。在冰天雪地里侦查突袭,斗智斗勇,越山跨谷,打虎上山。

读《林海雪原》,经常会感觉到一个词的含义,那便是“集体”。文中的人物,如少剑波杨荣等,张口闭口不离集体,可谓已经深入骨髓,同自己的生命融为—体了。风里来,雪里去,是记忆拼搏下的一点一滴。“集体”这个词,更相信无人不知无人不晓,它的意境已将不可计数的心融为一体,有着强烈的集体思维的人,头可断,血可留,此志不可改,永远把集体利益摆在第—位。

正是他们心中一直存在的对集体的坚定的信念支撑着他们。支持他们在风雪中一次次站起,支持他们在困难面前一次次昂首。他们不断追求着自己的人生目标,不畏艰难险阻,哪怕是到了死亡的边沿,也甘愿拼死一搏。也许他们认为唯有拼过,才了无遗憾。一个人如果没有集体的概念,没有理想,在自己的人生路上便会感到迷茫,徘徊不定,在黑暗中就会逐渐消逝。

而新时期,作为一名大学生,我们也要有自己的奋斗目标,要有坚强的意志,勇往直前的精神和大局的集体意识。这样才不会被现实击倒。理想是人生路上的明灯,是人生的彼岸。所以,我们要每时每刻心怀目标,从小目标到大目标,一步一步前进,一步一步迈向成功,不要让自己的人生留下遗憾。集体是个人得到升华的地方,纯粹为自己而活的人,失去了对集体、对社会的贡献之后,也就失去了维系人与社会之间的纽带,失去了社会性,也就容易随波逐流,迷失在物欲横流之中。

《林海雪原》,意犹未尽地读了好几遍。这些人物一个个机智勇敢,特别是杨子荣,他在被身份揭穿的同时还临危不惧,最后反败为胜,处决了小炉匠。在凶恶的敌人面前,依然十分镇定。这些小分队勇往直前的精神令我十分佩服。

《林海雪原》,讲述的不仅是解放战争初期的剿匪斗争,它所表现出的更是一种智慧,一股勇气,一分人性的美,融合为人类近乎完美的形象,成为一个世界的缩影,化为一片蓝天。
\end{document}
