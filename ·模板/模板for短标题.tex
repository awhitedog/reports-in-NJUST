\documentclass[12pt]{article}

\usepackage[UTF8]{ctex}
\usepackage{appendix}
\usepackage{enumerate}
\usepackage{amsmath}
\usepackage{graphicx}
\usepackage{cite}
\usepackage{array}
\usepackage{bigstrut}
\usepackage{geometry}
\geometry{left =2.5 cm,right=2.5cm,top=2.5cm,bottom=2.5cm}
\usepackage{multirow}
\usepackage{lastpage}
\usepackage{longtable}
\usepackage{listings}
 \usepackage{color}
  \usepackage{xcolor}
  \lstset{
  language=Matlab,  %代码语言使用的是matlab
  rulesepcolor=\color{red!20!green!20!blue!20},%代码块边框为淡青色
  keywordstyle=\color{blue!90}\bfseries, %代码关键字的颜色为蓝色,粗体
    numbers=left, % 显示行号
    numberstyle=\tiny,    % 行号字体
   commentstyle=\color[RGB]{0,130,0},    % 设置代码注释的颜色
  showstringspaces=false,%不显示代码字符串中间的空格标记
  stringstyle=\ttfamily, % 代码字符串的特殊格式
  breaklines=true, %对过长的代码自动换行
  extendedchars=false,  %解决代码跨页时,章节标题,页眉等汉字不显示的问题
  escapeinside=``,      % 代码中出现中文必须加上,否则报错
  texcl=true,}
\usepackage[section]{placeins}
\usepackage[colorlinks,linkcolor=blue]{hyperref}
\usepackage{titlesec}
\usepackage{titletoc}
\titleformat{\section}{\centering\heiti\zihao{3}}{\thesection}{0.3em}{}
\titleformat{\subsection}{\heiti \fontsize{12pt}{0}}{\thesubsection}{0.3em}{}
\renewcommand\figurename{\heiti\zihao{5} 图}
\renewcommand\tablename{\heiti\zihao{5} 表}
\renewcommand {\thetable} {\thesection{}.\arabic{table}}
\renewcommand {\thefigure} {\thesection{}.\arabic{figure}}
\date{}
\geometry{a4paper,scale=0.8}

\begin{document}%文档从这里开始。

\numberwithin{footnote}{section}
\renewcommand{\contentsname}{\centering 目录}
\renewcommand{\tablename}{表}
\renewcommand{\figurename}{图}
\renewcommand\refname{参考文献}
\renewcommand\appendix{\setcounter{secnumdepth}{0}}
\renewcommand\abstractname{摘要}
\begin{figure}[h]
  \centering
  \includegraphics[width=.6\textwidth]{logo}
\end{figure}
\thispagestyle{empty}
\begin{center}
\begin{songti}
\zihao{0}\textbf{高频电子线路综合实验}\\
\zihao{0}\textbf{实验报告}\\\ \\\
\zihao{3}
\\ \
\renewcommand\arraystretch{1.5}
\begin{tabular}{p{1.7cm}<{\centering}p{0.2cm}<{\centering}p{3.6cm}<{\centering}p{1.7cm}<{\centering}p{0.2cm}<{\centering}p{3.5cm}<{\centering}}
作\ \ 者&\textbf{:}&许晓明&学\  \ 号&\textbf{:}&9161040G0734\\\cline{3-3}\cline{6-6}
学\  \ 院&\textbf{:}&\multicolumn{4}{c}{电光学院}\\\cline{3-6}
专\ \ 业&\textbf{:}&\multicolumn{4}{c}{电子信息工程}\\\cline{3-6}
班\ \ 级&\textbf{:}&\multicolumn{4}{c}{电信3班}\\\cline{3-6}
题 \ \ 目&\textbf{:}&\multicolumn{4}{c}{EDA设计(II):}\\\cline{3-6}
&&\multicolumn{4}{c}{多功能数字钟设计}\\\cline{3-6}
指导者&\textbf{:}&\multicolumn{4}{c}{***}\\\cline{3-6}
\end{tabular}
\end{songti}
\end{center}
\begin{table}[b]
  \centering\zihao{3}
\number\year\ 年\ \number\month月
\end{table}

\begin{center}
\newpage
%\raggedright
\zihao{4}
\newpage
\pagenumbering{Roman}
\setcounter{page}{1}
\tableofcontents
\newpage
\pagenumbering{arabic}
\setcounter{page}{1}

\end{center}

\setcounter{page}{1}
\begin{table}
  \centering
  \begin{tabular}{|c|c|}
    \hline
    % after \\: \hline or \cline{col1-col2} \cline{col3-col4} ...
    1 & 2 \\
    3 & 1 \\
    \hline
  \end{tabular}
  \caption{}\label{123}
\end{table}
\section{设计要求说明}
\setcounter{equation}{0}
\autoref{123}
\setcounter{table}{0}
\setcounter{figure}{0}
\verb|\begin{document}|
2
3 \begin{verbatim}
4 在这个环境中所有的命令都不起作用。\end{document}
5 \end{verbatim}
\end{document}
